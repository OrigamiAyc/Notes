\documentclass[]{report}
\usepackage[hmargin=1.25in,vmargin=1in]{geometry} %调整页边距
% \usepackage[inner=1in,outer=1.25in]{geometry} %书籍左右不等宽排版
\usepackage[utf8]{inputenc}
\usepackage[]{ctex} %据说可以直接调用诸如 \kaishu \fangsong \heiti 的命令修改字体
\usepackage[svgnames]{xcolor} % Using colors
% \usepackage{background} % To include background images
\usepackage{fancyhdr} % Needed to define custom headers/footers
\usepackage[]{xeCJK}
\setCJKmainfont[BoldFont = STHeiti, ItalicFont = STKaiti]{Songti SC Light} %中文主字体
\setCJKsansfont[BoldFont = Weibei SC, ItalicFont = HanziPen SC]{Xingkai SC Light} %中文无衬线字体
\setCJKmonofont[BoldFont = Libian SC, ItalicFont = STFangsong]{Yuanti SC Light} %中文等宽字体
\setmainfont{Times New Roman} %\rmfamily
\setsansfont[ItalicFont = American Typewriter]{Comic Sans MS} %\sffamily
\setmonofont{Courier} %\ttfamily
\newfontfamily\monaco{Courier} % 用于代码段字体设置
\usepackage{titlesec}
\titleformat{\chapter}{\centering\huge\bfseries}{第~\thechapter~章}{1em}{}
\titleformat{\section}{\Large\bfseries}{第~\thesection~节}{1em}{}
\usepackage{lipsum} %填充文本

\usepackage{ulem} %解决下划线、删除线之类的
\usepackage{listings}
\lstset{
language=C++,
numberstyle = \monaco,
basicstyle = \monaco,
keywordstyle = \color{blue}\bfseries,
commentstyle=\color[HTML]{006400},
tabsize = 4,
%backgroundcolor=\color{bg}
emph = {int,float,double,char},emphstyle=\color{cyan},
emph = {[2]const, typedef},emphstyle = {[2]\color{red}} }

\makeatletter
\newif\if@restonecol
\makeatother
\let\algorithm\relax
\let\endalgorithm\relax
\usepackage[linesnumbered,ruled,vlined]{algorithm2e}%[ruled,vlined]{
\usepackage{algpseudocode}
\usepackage{amsmath}
\renewcommand{\algorithmicrequire}{\textbf{Input:}}  % Use Input in the format of Algorithm
\renewcommand{\algorithmicensure}{\textbf{Output:}} % Use Output in the format of Algorithm

\usepackage{amsmath} %数学公式问题
\usepackage{amsthm} %公式环境,如proof
\usepackage{booktabs} %三线表
\newcommand{\tabincell}[2]{\begin{tabular}{@{}#1@{}}#2\end{tabular}} %解决单元格内部换行的问题
% 比如这个 Beijing & 0,5 & 1,6 & 2,7 & 3,8 & 4,9 & The number changes every 3 months \\
% 改成这个 \tabincell{l}{Beijing}& \tabincell{c}{0,5}& \tabincell{c}{1,6}& \tabincell{c}{2,7}& \tabincell{c}{3,8}& \tabincell{c}{4,9}& \tabincell{c}{The number changes \\ every 3 months} \\
% 一个单元格过长,整行都需要修改
% 可以配合 \resizebox*{h-width}{v-width}{contents, e.g.tabular} 使用

\usepackage{mathrsfs} %在公式里面使用那个最花的字体
\usepackage{amssymb} %公式里面用空心黑体和旧式字体
\usepackage{amssymb} %AMS符号
\usepackage{amsthm} %AMS定理环境

\usepackage{markdown} %使用markdown语法,在编译时需要打开 shell-escape 标记,即 $ xelatex --shell-escape example.tex
\markdownSetup{hashEnumerators = true} %允许使用 #. 的方式编写有序列表
\markdownSetup{inlineFootnotes = true} %允许使用脚注形式的超链接,调用语法为 [anchor](uri), ^[footnote], <uri>
\markdownSetup{fencedCode = true} %以反引号和缩进来插入代码段,相当于 verbatim
\markdownSetup{
  pipeTables = true
} %支持表格的用法 (图片已经在markdown包里面支持了)
% \usepackage{booktabs} %解决三线表的线条粗细问题

\usepackage{graphicx} %插入图片
\usepackage{pdfpages} %插入PDF文件
\usepackage{makeidx}

\usepackage{tikz} %带圈字符
\usepackage{etoolbox} %带圈字符 (提供robustify)
\usepackage{enumitem}
\newcommand*{\circled}[1]{\lower.7ex\hbox{\tikz\draw (0pt, 0pt)%
    circle (.5em) node {\makebox[1em][c]{\small #1}};}} %新定义命令:带圈字符
\robustify{\circled}
% \usepackage{enumerate} %有序列表

\usepackage{hyperref} %超链接
% \usepackage[hidelinks]{hyperref} %隐藏超链接的红框
\markdownSetup{
  inlineFootnotes = true,
  renderers = {
    link = {\href{#3}{#1}},
  }
} % markdown块中使用直接点进去的超链接
% \setlist[enumerate,1]{label=(\arabic*).,font=\textup,leftmargin=7mm,labelsep=1.5mm,topsep=0mm,itemsep=-0.8mm}
% \setlist[enumerate,2]{label=(\alph*).,font=\textup,leftmargin=7mm,labelsep=1.5mm,topsep=-0.8mm,itemsep=-0.8mm}

\usepackage{braket}

%%%%%% Setting up the style

% \setlength\parindent{0pt} % Gets rid of all indentation
% \backgroundsetup{contents={\includegraphics[width=\textwidth]{ustc-name.pdf}},scale=0.4,placement=top,opacity=0.6,color=cyan,vshift=-20pt} %  USTC Logo

\pagestyle{fancy} % Enables the custom headers/footers

% 使用默认的Chapter页眉
% \lhead{} \rhead{} % Headers - all  empty

% \title{\vspace{-1.8cm}  \color{DarkRed} Laboratory Rotation Report}
% \subtitle{Title of the proposal % Title of the rotation project
% \vspace{-2cm} }
% \date{\today} % No date

\lfoot{\color{Grey} \textit{上官凝}}  % Write your name here
\rfoot{ \color{Grey} 编译原理 }
\cfoot{\color{Grey} \thepage}

\renewcommand{\headrulewidth}{0.0pt} % No header rule
\renewcommand{\footrulewidth}{0.4pt} % Thin footer rule

\title{{\huge {编译原理笔记}}}
\author{上官凝}
\date{\today}

\linespread{1.3} %行间距为1.3倍默认间距 (1.3 x 1.2倍字符宽度)

\makeindex

\begin{document}
\theoremstyle{definition} \newtheorem{theorem}{Thm}[section] %定义一个定理Thm,序号为section的下一级序号
\theoremstyle{definition} \newtheorem{definition}{Def}[section] %定义一个定义Def,序号为section的下一级序号
\theoremstyle{plain} \newtheorem{lemma}{lemma}[section] %引理

	\maketitle
	\newpage

	\tableofcontents
	\newpage

	\chapter{语法分析}
	\textit{输入:词法分析获得的记号序列;输出:程序的语法树}。需要一种可以描述合法记号序列的语言、一种可以区分合法和非法的记号序列的方法。正则表达式不能用于描述配对或嵌套的结构,这是因为有限自动机不能记录重复访问同一状态的次数。\par
	\section{上下文无关文法}
		上下文无关文法CFG是由一个四元组$(V_T,V_N,S,P)$组成的:
		\begin{enumerate}
			\item 终结符:就是“句型”最终需要化成的形式,里面的元素都是确定的符号,也称为$token$。\textit{在谈论编程语言的文法时,记号名时终结符的同义词。}
			\item 非终结符:可以认为是推导过程中的“中间变量”,或“形式参量”,用自下而上的角度来说,非终结符是终结符或/和非终结符的规约
			\item 开始符号:是语法树的$root$
			\item 产生式的有限集合:生成规则,用$\to$
		\end{enumerate}
		\subsection{CFG推导}
		CFG推导是“从文法推出文法所描述的语言中所包含的合法串集合的动作”,也就是说,从开始符号开始,利用生成规则进行迭代替换,得到句型。“上下文无关”是指在推导过程中,每一步$\alpha A\beta\Rightarrow\alpha\gamma\beta$仅依赖于$A\to\gamma$,而与$\alpha,\beta$无关。对于语言、文法、句型、句子,语言是句子的集合,句子是实例。最左推导$\Rightarrow_{lm}$和最右推导(规范推导)$\Rightarrow_{rm}$。一步推导$\Rightarrow$和零步或多步推导$\Rightarrow^*$以及一步或多步推导$\Rightarrow^+$。\par
		\subsection{文法的二义性}
		文法的某些句子存在不止一种最左(最右)推导,或者不止一棵分析树,则该文法是二义的。一般是由优先级和结合性的不确定导致的。所以可以通过定义优先级(将多个选项拆分为不同非终结符的嵌套推导规则来认为定义等级划分)来解决这个问题。这样,更接近于开始符号的非终结符就不能直接推导到终结符,更接近于终结符的不能从开始符号终结推导。\textit{左推导优先级从高到低,右推导相反}比如将第一个式子换成下面的几个:
		\[\begin{gathered}
			E\to E+E\mid E*E\mid(E)\mid\mathbf{id}\\
			E\to E+E\mid F\\
			F\to F*F\mid G\\
			G\to (G)\mid\mathbf{id}
		\end{gathered}\]
		\subsection{悬空else文法}
		悬空else文法是一个经典的例子:
		\begin{figure}[h!]
			\centering
			\begin{minipage}{20em}
				\centering
				\includegraphics[scale = 0.2]{images/Impending_ELSE.pdf}
				\caption{悬空else文法及其二义性}
			\end{minipage}\quad
			\begin{minipage}{20em}
				\centering
				\includegraphics[scale = 0.2]{images/Eliminate_Ambiguity_of_ELSE.pdf}
				\caption{悬空else文法二义性的消除}
			\end{minipage}
		\end{figure}\par
		应当注意的是,文法的二义性并不意味着语言是二义的。只有当产生一个语言的所有文法都是二义的时,才称为是二义的。而且也可以构造允许二义文法的分析器,但要附带消除二义性但规则(剪枝回溯)。值得看到的是,定义无二义文法可能会失去简洁性。
		\subsection{左递归}
		一个文法是左递归的,如果他有非终结符A,对于某个串$\alpha$,存在推导$A\Rightarrow^+A\alpha$。自上而下的分析方法不能用于左递归文法。
	\section{自顶向下分析方法}
		\subsection{递归下降分析方法}
		包括一个输入缓冲区和向前看指针lookahead,自左向右扫描输入串;设计一个辅助过程match(),将lookahead指向的位置与产生式迭代生成的终结符进行匹配,如匹配,将lookahead挪到下一个位置。就是对于一个可能的串,从root开始构建一棵语法的生成树。递归下降也有着自己的一些问题,比如可能进入无限循环。下面是递归下降的三个问题,
		\subsection{消除左递归}
		左递归分为直接左递归和间接左递归。直接左递归有形如$A\rightarrow A\alpha$的产生式,间接左递归是通过有限次迭代(类似于证明序列)得到(非直接左递归首先变成直接左递归,然后消除他)。消除左递归的通用方法如下所示:首先将A的产生式组合在一起:
		\[A\to A\alpha_1\mid A\alpha_2\mid\cdots\mid A\alpha_m\mid\beta_1\mid\beta_2\mid\cdots\mid\beta_n\]
		其中$\beta_i$都不以A开始,$\alpha_i$非空,然后用
		\[\begin{gathered}
			A\to\beta_1A'\mid\beta_2A'\mid\cdots\mid\beta_nA'\\
			A'\to\alpha_1A'\mid\alpha_2A'\mid\cdots\mid\alpha_mA'\mid\varepsilon
		\end{gathered}\]
		代替A的产生式。注意后面加上了一个$\varepsilon$,作为一个终结符结束递归
		\subsection{有左因子的文法}
		有左因子的(left-factored)文法形如$A\to\alpha\beta_1\mid\alpha\beta_2$。在自上而下的分析中,当不清楚应该用非终结符$A$的那个选择来替换它时,可以通过重写$A$产生式来推迟这种决定,推迟到看到足够多的输入,能帮助正确决定所需选择为止。如上式等价于\[\begin{gathered}
			A\to\alpha A'\\
			A\to\beta_1\mid\beta_2
		\end{gathered}\]
		需要注意的是,提取左因子的重写并不能解决二义性,比如之前提到的悬空else文法,提取左因子之后成为
		\[\begin{gathered}
			stmt\to\mathbf{if}\ expr\ \mathbf{then}\ stmt\ optional_else_part\mid\mathbf{other}\\
			optional_else_part\to\mathbf{else}\ stmt\mid\varepsilon
		\end{gathered}\]
		这个文法对于如$\mathbf{if}\ expr\ \mathbf{then}\ \mathbf{if}\ expr\ \mathbf{then}\ stmt\ \mathbf{else}\ stmt$这个表达式时,有如下两种最左推导的分析树
		\begin{figure}[h!]
			\centering
			\begin{minipage}{20em}
				\centering
				\includegraphics[scale = 0.05]{images/Left_Ambiguity_1.JPG}
			\end{minipage}
			\begin{minipage}{20em}
				\centering
				\includegraphics[scale = 0.048]{images/Left_Ambiguity_2.JPG}
			\end{minipage}
		\end{figure}
		\subsection{复杂的回溯代价太高}
		非终结符有可能有多个产生式,由于信息缺失,无法准确预测选择哪一个,考虑到往往需要对多个非终结符进行推导展开,因此尝试的路径可能呈指数级爆炸
	\section{预测分析法}
		Predictive Parsing与递归下降法类似,但是没有若干尝试,没有回溯,采用的通过向前看一些记号来预测需要用到的产生式的方法。此方法接收LL(k)文法,即自左向右读取(left-to-right),最左推导(leftmost derivation),基于对k个记号向前看的推导。在实际应用中,LL(1)用的最广泛。
		\subsection{“第一个”集合和“跟随”集合}
		LL(1)文法中有两个最重要的集合,即FIRST($\alpha$)和FOLLOW($\alpha$)。
		\[FIRST(\alpha)=\{a\mid\alpha\Rightarrow^*a\dots\}\quad\alpha\in V_N,a\in V_T\]
		即,可以从$\alpha$推导出的句型的首部词的集合
		\[FOLLOW(A)=\{a\mid S\Rightarrow^*\dots Aa\dots\}\quad\alpha\in V_N,a\in V_T\]
		即,可能在推导过程中紧跟在A右边的终结符号的集合
		\begin{figure}[h!]
			\centering
			\begin{minipage}{20em}
				\centering
				\includegraphics[scale = 0.2]{images/LL1_first.pdf}
				\caption{确定给定推导文法的FIRST集合}
			\end{minipage}
			\begin{minipage}{20em}
				\centering
				\includegraphics[scale = 0.2]{images/LL1_follow.pdf}
				\caption{确定给定推导文法的FOLLOW集合}
			\end{minipage}
		\end{figure}
		\subsection{LL(1)文法}
		则有LL(1)文法的定义为:对于任何两个产生式$A\to\alpha\mid\beta$都有
		\begin{enumerate}
			\item $FIRST(\alpha)\cap FIRST(\beta)=\varnothing$
			\item 若$\beta\Rightarrow^*\varepsilon$,那么$FOLLOW(A)\cap FIRST(\alpha)=\varnothing$
		\end{enumerate}
		对于第二条,假设$FIRST(A)\cap FIRST(\alpha)=\{a\}$,即
		\[\begin{gathered}
			S\Rightarrow^*\dots Aa\dots\\
			A\Rightarrow^*a\alpha'
		\end{gathered}\]
		又由于$\beta\Rightarrow^*\varepsilon$所以当遇到$Aa$时,既可以用$A\to\alpha$来展开,也可以用$A\to\beta,\beta\Rightarrow^*\varepsilon$来消去A。\par
		LL(1)文法有一些明显的性质,如没有公共左因子(提取左因子)、不是二义的(规则重写)、不含左递归(化为直接左递归,拆分)
		\section{递归下降的预测分析}
		所谓预测分析是指能根据当前的输入符号为非终结符确定采用哪一个选择(预读取k个token)。在这一个部分,主要是要求写出一个预测分析的程序伪代码。伪代码主要是包括两个部分,即$match$函数和对于每一个非终结符的函数。以如下的文法为例:
		\[\begin{gathered}
			type\to simple\mid\uparrow\mathbf{id}\mid\mathbf{array}\ [simple]\ \mathbf{of}\ type\\
			simple\to\mathbf{integer}\mid\mathbf{char}\mid\mathbf{num\ dotdot\ num}
		\end{gathered}\]
		其递归下降预测分析器(伪代码)为(部分):
		\begin{lstlisting}
void match(terminal t)
{
	if (lookahead == t) lookahead = nextToken();
	else error();
}

void type()
{
	if ((lookahead == integer) || (lookahead == char)
		|| (lookahead == num))
		simple();
	else if (lookahead == '\uparrow')
	{
		match ('\uparrow');
		match (id);
	}
	else if (lookahead == array)
	{
		/* code */
	}
	else error();
}
		\end{lstlisting}
		具体来说,就是每一个非终结符$T$,需要构造过程$void T()$,在其中对$T$的产生式右端$\alpha$进行匹配。在匹配的判断中使用$FIRST$集合。匹配之后执行的动作为:非终结符调用他自己的过程,终结符$a$调用$match(a)$
		\subsection{预测分析表}
		对文法的每一个产生式$A\to\alpha$,执行以下两个步骤:
		\begin{enumerate}
			\item 对$FIRST(\alpha)$的每个终结符$A$,把$A\to\alpha$ 加入$M[A, a]$,注意这里的$\alpha$是一个表达式
			\item 如果$\varepsilon$在$FIRST(\alpha)$中,$对FOLLOW(A)$的每个终结符$b$(包括$\$$), 把$A\to\alpha$加入$M[A, b]$
		\end{enumerate}
		值得注意的是,填入表格的$A\to\alpha$是一条推导文法,如$E\to TE'$或者$T'\to\varepsilon$
		注:多重定义条目意味着文法左递归或者是二义的,但是可以通过删除部分条目来消除
	\section{自底向上分析方法}
		\textbf{自左向右读取!}基本方法是归约(右推导的逆过程),用到句柄(可归约串,可能不唯一)的概念,思路是针对输入串,尝试根据产生式规则归约(reduce)到文法的开始符号,是一个比自顶向下更一般化的方法。归约是每一步,特定子串被替换为相匹配的某个产生式左部的非终结符。\par
		对于句柄,句柄是“该句型中和某产生式右部匹配的子串,并且把它归约成该产生式左部的非终结符,代表了最右推导的逆过程的一步”。首先句柄是字串而不一定是一个完整的串,它是在归约过程中某一步产生的句型中的一部分。从另一个方向来看,句柄是对应者自顶向下推导的过程中分析树的非树叶的结点。句柄的右边\textbf{仅包含终结符}(这是因为归约是最右推导的逆过程)。这里“句柄的右边”指的是在这一步规约的句型中,位置在此句柄右边的串。如果文法二义,那么句柄可能不唯一。\par
		比如说,在如下的最右推导(逆过程就是归约)中:
		\[\begin{gathered}
			S\to aABe\\
			A\to Abc\mid b\\
			B\to d\\
			S\Rightarrow_{rm}aABe\Rightarrow_{rm}aAde\Rightarrow_{rm}aAbcde\Rightarrow_{rm}abbcde
		\end{gathered}\]\par
		在句型$aAbcde$中,这一步用于规约的句柄为$Abc$,即“句型中的字串”,同时为产生式$A\to Abc$的右部匹配。
		\subsection{移进(shift)-归约(reduce)分析技术}
		\begin{enumerate}
			\item 两个空间:栈(用来保存已经扫描过的文法符号)、缓冲区(用来保存还未分析的文法符号)
			\item 四个状态:移进(shift,将下一个输入符号放到栈顶,以形成句柄)、归约(reduce,句柄替换为对应的产生式的左部非终结符)、接收(accept,分析成功)、报错(error)
		\end{enumerate}
		移进-归约技术需要解决一些问题,如:
		\begin{enumerate}
			\item 移进-归约冲突(如何决策选择移进(构造句柄)还是归约)
			\item 进行归约时,确定右句型中将要归约的子串(识别句柄)
			\item 归约-归约冲突(进行归约时,如何确定选择哪一个产生式)
		\end{enumerate}
	\section{LR(k)分析技术}
	【回顾】自顶向下和自底向上:\par
	\begin{enumerate}
		\item 自顶向下(Top-down)
		\begin{enumerate}
			\item 针对输入串,从文法的开始符号出发,尝试根据产生式规则推导(derive)出该输入串。
			\item LL(1)文法及非递归预测分析方法
			\item left-to-right scan + leftmost derivation
		\end{enumerate}
		\item 自底向上(Bottom-up)
		\begin{enumerate}
			\item 针对输入串,尝试根据产生式规则归约(reduce)到文法的开始符号。
			\item LR(k)文法及其分析器
			\item left-to-right scan + rightmost derivation
		\end{enumerate}
	\end{enumerate}\par
	一个文法,如果能为他构造出所有条目都唯一都LR分析表,就说他是LR文法。Say,LR语法分析器的关键在于构造LR分析表:\begin{enumerate}
		\item 计算所有可能的状态
		\begin{enumerate}
			\item 每一个状态描述了语法分析过程中所处的位置
			\item 可确定正在分析的产生式集合
			\item 可确定句柄形成的中间步骤
		\end{enumerate}
		\item 明确状态之前的跳转关系
		\item 明确状态与输入之间对应的移进或者归约操作
	\end{enumerate}\par
	LR分析器的格局是二元组,其第一个成分是栈的内容,第二个成分是尚未扫描的输入。活前缀是最右句型的前缀,该前缀不超过最右\textit{句柄}的右端。分析表的转移函数实际上是识别活前缀的DFA,规约函数是对句柄
	\section{SLR}
	文法G的LR(0)项目(简称项目)是在其右部的某个地方加点,表示分析过程的状态的产生式。点的左边是已经看见(读入)的部分,右边是期待看见的部分。或者说,左边是历史信息,右边是展望信息。\par
		\subsection{产生规则} 为了构造出指定文法的SLR分析表,首先重写此文法:
		\begin{enumerate}[label = \circled{\arabic{*}}]
			\item 增加一个新的开始符号(产生拓广文法),即如果之前文法G的开始符号是S,那么增加开始符号$S'$和产生式$S'\to S$,新增产生式的目的是用来指示分析器什么时候应该停止分析并宣布接收输入
			\item 将多路选择的产生式拆分开,比如$E\to E+T\mid T$拆分成$E\to E+T$和$E\to T$两个产生式,这样对判断状态集有好处
			\item 将所有的产生式编号,从$S'\to S\quad(0)$开始
		\end{enumerate}
		然后从新的开始产生式开始,用下面两条规则构造项目集的闭包:(和书上写的不一样)
		\begin{enumerate}
			\item 由上一个状态转换过来的产生式(点向后移动一个位置)
			\item 如果某一条产生式的点在一个非终结符前面,那么看这个非终结符的所有产生式,并将其初始产生式假如到闭包中
		\end{enumerate}
		解释一下上面的规则:
		\begin{enumerate}[label = \circled{\arabic{*}}]
			\item 我自己定义一个概念叫\textit{初始产生式},意思是点加在产生式右部的最左端。(其实就是书上P76写的非核心项目加上$S'\to\cdot S$,如产生式$E\to E+T$,其初始产生式就是$E\to\cdot\, E+T$
			\item 在生成状态$I_0$时,从拓展文法的开始产生式$S'\to S$的初始产生式$S'\to \cdot S$开始
			\item 状态之间的跳转并不需要对前状态的所有元素(产生式)都满足,只表达一种对于此项目集\textbf{可能的}跳转模式
			\item 跳转状态,意味着接收了某种输入
			\item 点在不同位置的同一产生式视作两个不同的项目,需要单独加进去。比如本来我有一个产生式$E\to TE$,现在在闭包里面有$E\to T\cdot E$,我还要加进去$E\to \cdot TE$
		\end{enumerate}
		\subsection{有效项目}
		如果说$S'\Rightarrow^*\alpha Aw\Rightarrow\alpha\beta_1\beta_2 w$,那么就说项目$A\to\beta_1\beta_2$对活前缀$\alpha\beta_1$是有效的。一个项目对于好几个活前缀都可以是有效的,比如若$\beta_2\neq\varepsilon$则应当移进(继续读取才能规约为$A$);若$\beta_2=\varepsilon$,则已经可以直接归约了。
		\subsection{SLR分析表}
		SLR分析表分为左右两部分,左边是动作(action)表,右边是转移(goto)表。二者的行标分别为终结符加$\$$和非终结符,列标为项目集(状态)。action表分为三种情况:
		\begin{enumerate}
			\item 对于没有读取完的产生式项目(点在右部中间,形如$A\to\alpha\cdot a\beta$,$a$为终结符,$\alpha,\beta$可以为$\varepsilon$),设这个项目所在的状态为$I_i$,读取$a$之后会跳转到$I_j$,即$I_i\stackrel{a}{\to}I_j$,那么就在分析表的第i行第a列填入$s_j$
			\item 对于读取完毕的产生式项目,设这个项目所在的状态为$I_i$,产生式的编号为$j$,那么就在分析表的第i行第a列填入$r_j$,其中a是FOLLOW(A)中的所有元素
			\item 对于包含项目$S'\to S\cdot$的状态$i$,在第i行第$\$$列填入$acc$
		\end{enumerate}
		转移表就是说如果状态i在接收非终结符A之后转移到状态j,那么就在第i行第A列填入j
		\subsection{SLR(1)文法}
		一个上下文无关文法G,通过上述算法构造出SLR语法分析表,且表项中没有移进/归约或 者归约/归约冲突,那么G就是SLR(1)文法。1代表了当看到某个产生式右部时,只需要再向前看1个符号就可决定是否用该式进行归约。通常可以省略1,写作SLR文法。
		\subsection{判定满足SLR文法输入串}
		根据SLR文法分析表,使用“文法符号栈-输入缓冲区-对应行为”三联表的形式,尝试读取并分析输入串,直到ACC或者ERR
	\section{规范的LR分析方法}
	在识别活前缀DFA的状态中,增加信息,可以排除一些不正确的归约操作,故规范的LR分析方法增加了前向搜索符:一个项目$A\to\alpha\cdot\beta$,如果真的用这个产生式进行规约之后,期望看到的符号是a(换句话说,在自底向上分析的过程中,对产生式右部$\alpha\beta$进行规约为$A$之后,在$A$的右边应当出现的符号)。LR(1)项目是一个二元组(SLR中的项,搜索符),形式的定义为$[A\to\alpha\cdot\beta,a]$。当$\beta$不为空的时候,a不起作用,当$\beta$为空的时候,如果下一个输入符号为a,那么将按照$A\to\alpha$进行规约,故有$a$的集合是FOLLOW(A)的子集。\par
	\begin{definition}
		对活前缀有效\par
		称LR(1)项目$[A\to\alpha\cdot\beta,a]$对活前缀$\gamma$有效,当且仅当如下情形:如果存在着推导$S'\Rightarrow^*_{rm}\delta Aw\Rightarrow_{rm}\delta\alpha\beta w$,其中$\gamma=\delta\alpha$,a是w的第一个符号,或者a是\$且w是$\varepsilon$
	\end{definition}
		\subsection{构造LR(1)项目集规范族}
		先声明一个我自己瞎起的名字:产生式左边的叫\textit{左值},右边的都叫\textit{右值}。构造方法和SLR的构造方法类似:
		\begin{enumerate}
			\item 不妨设拓广文法的新开始产生式为$S'\to S$,那么从$S'\to\cdot S,\$$开始
			\item 【扩充闭包】在一个状态(项目集)中,对他现有的每一个项目$[A\to\alpha\cdot B\beta,a]$,进行如下构造:
			\begin{enumerate}
				\item 【左值】考虑点后面的非终结符B
				\item 【右值】对于拓展文法中的所有B在左边的产生式$B\to\gamma$(和SLR考虑的一样)
				\item 【搜索符】考虑$FIRST(\beta a)$中的每一个终结符b,并将$[B\to\cdots\gamma,b]$不重复地加入到状态集合中
			\end{enumerate}
			\item 【找全状态】规则和SLR一样,只是要注意,不仅点的位置不同就不同,搜索符不同也不同
		\end{enumerate}
		回顾一下FIRST集合的确定
		\begin{figure}[h!]
			\centering
			\begin{minipage}{20em}
				\centering
				\includegraphics[scale = 0.2]{images/LL1_first.pdf}
				\caption{确定给定推导文法的FIRST集合}
			\end{minipage}
			\begin{minipage}{20em}
				\centering
				\includegraphics[scale = 0.2]{images/LL1_follow.pdf}
				\caption{确定给定推导文法的FOLLOW集合}
			\end{minipage}
		\end{figure}
		\subsection{构造规范的LR分析表}
		表格形式与SLR类似,但是规则稍有变动:
		\begin{enumerate}
			\item 当点在右值中间时($[A\to\alpha\cdot a\beta,b]$),根据$I_i\stackrel{a}{\to}I_j$,将$s_j$填入i行a列。\textit{注意这种情况下,搜索符b是没有用的。}在这种产生式的形式下,b的唯一用处是在求闭包的时候,求$FIRST(\beta a)$
			\item 当点在右值最右端时,\textbf{不再看FOLLOW(A)},而是对于$I_i$中的所有$[A\to\alpha\cdot,a]$,将$r_j$(j为产生式编号)填入i行a列(其实就是看了FOLLOW的一个子集)
			\item acc和goto不变
		\end{enumerate}
		\subsection{LALR}
		就是在规范LR的基础上合并同心项目集($i.e.$略去搜索符之后他们是相同的集合)。并按照规范LR的规则构造LALR分析表。合并同心项目集可能会引起冲突,但不会引起新的移进-归约冲突

	\chapter{语法制导的翻译}
	这一章开始赋予抽象的符号系统以实际的含义,进入到生成中间代码的阶段。\par
	语法和文法就像数理逻辑中的(语法)和(语义),编译原理的语法是对前面文法分析树在具体程序语言(前提集)上的“赋值”,使得终结符和非终结符有了自己的“属性”。做一组不是很恰当的类比:语法=语法,语义规则=赋值,串=逻辑式,翻译=逻辑式的真假,文法符号=个体变元,属性=指派
	\section{语法制导的定义}
	语法制导(SDD)的定义是带属性和语义规则的上下文无关文法。SDD为CFG中的文法符号设置语义属性,用来表示语法成分对应的语义信息。语法制导翻译是使用上下文无关文法(CFG)来引导对语言的翻译,是一种面向文法的翻译技术。
	\paragraph{如何表示语法信息} 为CFG中的文法符号设置语义属性,用来表示语法成分对应的语义信息
	\paragraph{如何计算语义属性} 文法符号的语义属性值是用与文法符号所在产生式(语法规则)相关联的语义规则来计算的:对于给定的输入串x,构建x的语法分析树,并利用与产生式相关联的语义规则来计算分析树中各结点对应的语义属性值\par
	在语法制导中,用的是基础的上下文无关文法,每一个文法符号有一组属性,每一个文法产生式$A\to\alpha$有一组形式如$b=f(c_1,c_2,\cdots,c_k)$的语义规则,不一定这个产生式中的每一个元素的(每一个)属性都需要写进去。如果b是左值的属性,$c_1,\cdots,c_k$是右值的属性或者左值的其他属性,那么就叫b为综合属性。\textbf{终结符只能有综合属性,属性值无需计算,由词法分析给定}
		\subsection{综合属性和继承属性}
		仅仅使用综合属性地语法制导定义称为S属性定义。但是不是所有情况都这么简单,这时候就会用到继承属性(比如在消除了左递归的文法里面)\par
		总的来说,综合属性将属性自底向上传,而继承属性自上而下传。而对于如下产生式
		\[\begin{gathered}
			T\to FT'\\
			T'\to *FT'_1
		\end{gathered}\]
		在计算T的综合属性$T.syn$时,由于乘法运算符和操作数都隐藏在了$T'$里面,而这个隐藏内容在后续产生式$T'\to*FT'_1$里面,那么就需要将F的属性保存于T'的继承属性里面,传递到下一个“计算周期”,得到结果之后再返回到上一级并传递给T。如下:
		\begin{table}[h!]
			\centering
			\begin{tabular}{cc}
				\toprule
				产生式&语义规则\\
				\midrule
				$T\to FT'$&$T'.inh=F.val,\ T.val=T'.syn$\\
				$T'\to*FT'_1$&$T'_1.inh=T'.inh\times F.val,\ T'.syn=T'_1.syn$\\
				\bottomrule
			\end{tabular}
		\end{table}
		\subsection{注释分析树和属性依赖图}
		根据语法分析,构建出来语法分析树,然后在每一个节点根据其属性值进行计算。当这些属性都是综合属性时,计算可以自底向上地完成。\par
		如果分析树上的一个节点的属性b依赖于某个节点的属性c,那么b的语义规则的计算必须在c之后。这种依赖关系可以用一种有向图来描述。

\end{document}
