\documentclass[]{report}
\usepackage[hmargin=1.25in,vmargin=1in]{geometry} %调整页边距
% \usepackage[inner=1in,outer=1.25in]{geometry} %书籍左右不等宽排版
\usepackage[utf8]{inputenc}
% \usepackage[]{ctex} %据说可以直接调用诸如 \kaishu \fangsong \heiti 的命令修改字体
\usepackage[]{xeCJK}
\setCJKmainfont[BoldFont = STHeiti, ItalicFont = STKaiti]{Songti SC Light} %中文主字体
\setCJKsansfont[BoldFont = Weibei SC, ItalicFont = Xingkai SC Light]{Xingkai SC} %中文无衬线字体
\setCJKmonofont[BoldFont = Libian SC, ItalicFont = STFangsong]{Yuanti SC Light} %中文等宽字体
\setmainfont{Times New Roman} %\rmfamily
\setsansfont{Comic Sans MS} %\sffamily
\setmonofont{Courier} %\ttfamily

\usepackage{markdown} %使用markdown语法,在编译时需要打开 shell-escape 标记,即 $ xelatex --shell-escape example.tex
\markdownSetup{hashEnumerators = true} %允许使用 #. 的方式编写有序列表
\markdownSetup{inlineFootnotes = true} %允许使用脚注形式的超链接,调用语法为 [anchor](uri), ^[footnote], <uri>
\markdownSetup{fencedCode = true} %以反引号和缩进来插入代码段,相当于 verbatim
\markdownSetup{
  pipeTables = true
} %支持表格的用法 (图片已经在markdown包里面支持了)
\usepackage{booktabs} %解决三线表的线条粗细问题

\usepackage{graphicx} %插入图片
\usepackage{makeidx}

\usepackage{tikz} %带圈字符
\usepackage{etoolbox} %带圈字符 (提供robustify)
\usepackage{enumitem}
\newcommand*{\circled}[1]{\lower.7ex\hbox{\tikz\draw (0pt, 0pt)%
    circle (.5em) node {\makebox[1em][c]{\small #1}};}} %新定义命令:带圈字符
\robustify{\circled}
% \usepackage{enumerate} %有序列表

\usepackage{hyperref} %超链接
\markdownSetup{
  inlineFootnotes = true,
  renderers = {
    link = {\href{#3}{#1}},
  }
} % markdown块中使用直接点进去的超链接
% \setlist[enumerate,1]{label=(\arabic*).,font=\textup,leftmargin=7mm,labelsep=1.5mm,topsep=0mm,itemsep=-0.8mm}
% \setlist[enumerate,2]{label=(\alph*).,font=\textup,leftmargin=7mm,labelsep=1.5mm,topsep=-0.8mm,itemsep=-0.8mm}

\title{Summary for COD lab 6}
\author{Tobiichi Origami}
\date{\today}

\linespread{1.3} %行间距为1.3倍默认间距 (1.2倍字符宽度)

\makeindex

\begin{document}
	\maketitle


	\begin{center}
		\chapter[]{系统总线\footnote{唐书第三章}}
	\end{center}
		\section[]{总线标准\footnote{Ch 3.3.3}}\par
			\emph{所谓总线标准,可视为系统与各模块、模块与模块之间一个互联的标准界面。这个界面对它两端的模块都是透明的……按总线标准设计的接口都可以视为通用接口}\par
			重点关注:PCIe总线、USB总线、type-C线和lighting线?
		\section[]{总线结构\footnote{Ch 3.4}}\par
			参考这张图(其中的扩展总线接口用DMA)\par
			\includegraphics[scale = 0.25]{../pictures/Triple_Bus.png}\par
			所以需要设计一个cache(模块调用直接连到MEM段)
		\section[]{总线控制\footnote{Ch 3.5}}\par
			\emph{包括判优控制(或称仲裁逻辑)和通信控制}
			\subsection{总线判优控制}\par
				分为集中式和分布式两种:集中式又包括链式查询(会出现饿死的情况)、计数器定时查询 (可以通过计数器不清零来实现没有优先级)、独立请求 (硬件复杂)
			\subsection{总线通信控制}\par
				一次总线操作分为4个周期:申请分配、寻址、传数、结束 (对于仅有一个\textbf{主模块}的简单系统,无需申请、分配和撤除)。通信又分为同步和异步,同步通信有着固定的传输周期任务规定\footnote{唐书P60,电子书P72},并 (可能采用SYN标识开始);而异步通信采用握手的方式,分为不/半/全互锁三种 (即一/二/三次握手),对于传数的数据有打包处理 (一帧 = 1起始位(低电平) + 5~8个数据位+1个奇偶校验位 + 1/1.5/2个终止位(高电平)),任意两帧之间任意长度高电平。类似的还有半同步通信\footnote{唐书P64,电子书P76}和分离式通信。
				\paragraph{异步通信的应答} 异步通信的应答方式可分为不互锁、半互锁和全互锁三种类型:
				\subparagraph{不互锁}主模块(对总线有控制权的模块)发出请求信号之后,不必等待接到从模块的回答信号,而是经过一段时间自动撤销;从模块接到请求信号之后,\textit{在条件允许时发出回答信号},且经过一段时间自动撤销回答信号。例如,CPU向主存写入信息
				\subparagraph{半互锁}主模块发出请求信号,必须在\textit{等到从模块的回答信号}之后才能撤销请求信号;而从模块在接到请求信号之后发出应答信号,但是经过一段时间就直接撤销。例如,在多机系统中,某个CPU需访问共享存储器时,该CPU发出访存命令后,必须接收到存储器未被占用但回答信号,才可以进行访存
				\subparagraph{全互锁}双方都采用等待应答的方式。例如网络通信

	\begin{center}
		\chapter[]{输入输出系统\footnote{唐书第五章}}
	\end{center}
		\section{概述}
			主机和I/O设备之间联络的方式,分为:\par
			\begin{enumerate}[label = (\arabic{*})]
				\item 立即响应
				\item 异步工作采用应答信号联络,如前所述,用起始和终止信号来建立主从设备之间的联系
				\item 同步工作采用同步时标联络
			\end{enumerate}
			异步工作如图所示:如同\par
			\includegraphics[scale = 0.25]{../pictures/IO_Contaction.png}\par
			\subsection{I/O设备与主机信息传送的控制方式}\par
				\textbf{这个超级重点}\par
				共有5种方式:程序查询方式、程序中断方式、直接存储器存取方式 (DMA)、I/O通道方式、I/O处理机方式。主要用前三种,下面对前三种逐一整理\par
				\begin{enumerate}[label = (\arabic{*})]
					\item 程序查询方式\\当现行程序需要启动某I/O设备工作时,将查询程序插入到正在运行到程序中。只要一启动I/O设备,CPU就不断查询I/O设备的准备情况,从而终止了原程序的执行。这种方式相当于进程调度,context switch踢出去原有进程进入waiting状态,然后执行其所需的I/O进程 (貌似略去了ready状态)。\textit{查询通过做I/O指令完成:对I/O接口的某些控制寄存器置“0”或“1”,用于控制设备进行相关工作;测试设备的某些状态,如“忙”、“就绪”等,以便决定下一步操作;传送数据}
					\item 程序中断方式\\CPU在启动I/O设备后,继续执行现有程序,只是I/O设备准备就绪向CPU发出中断请求后才予以响应。这个方式是中断处理,需要保护现场 (当然前面那个也需要),详情见后面中断处理部分。程序中断方式不仅需要在硬件上增加相应的电路,而且在软件上面也需要编制相应的中断服务程序 (详见 5.3和5.5)
					\item DMA方式\\ (5.6)
				\end{enumerate}
			\textbf{关于程序中断方式等操作等一点Notes}\par
				{\itshape 注意这张图里面第三行的“尚未”应当去掉。意即在一次中断方式的I/O操作中,首先如果待传送数据过多,则拆分到不同的存取周期来传送 (可以采用并行传输,二者并不影响);其次在CPU需要用到I/O时,先启动并继续执行现有程序,等待设备的中断到达之后,开始执行I/O的取数据 (从接口的buffer经CPU传送至内存);在这个传输过程中,可能会发生诸如异常处理中断之类的,需要按照中断处理机制嵌套递归;否则待这一次传输全部结束之后,才开始后面的CPU指令执行 (可以采取关闭中断的措施,如互斥锁、MASK)}\par
				\includegraphics[scale = 0.25]{../pictures/recursive_interrupt.png}
		\section{I/O接口}
			使用接口的理由 (接口的作用):\\数据缓冲 (读取速度不一样)、串 - 并格式的转换 (CPU并行)、电平转换、传送 (CPU到I/O的) 控制命令、监视设备工作状态 (忙、就绪、错误、中断请求)。注意端口 (Port) 是接口 (Interface) 电路中的一些寄存器 (接口还包括控制逻辑电路),包括数据端口、控制端口、状态端口。接口与总线、设备的连接如图所示:\par
			\includegraphics[scale = 0.25]{../pictures/IO_Interafce.png}\par
			\includegraphics[scale = 0.25]{../pictures/IO_Interface_2.png}\par
			现代计算机大多采用三态逻辑电路来构成总线。\par\par
			接口的功能和组成:\\选址功能 (片选,设备选择线上的设备码与本设备的设备码相符时,发出设备选中信号SEL)、传送命令的功能 (设有命令寄存器(存放I/O指令中的指令码,受SEL控制写使能) 和命令译码器)、传送数据的功能 (有数据暂存寄存器)、反映I/O设备工作状态的功能 ( $^{eg.}$ 用完成触发器D和工作触发器B来标志设备所处的状态:D = B = 0为暂停状态;D = 1, B = 0为就绪状态;D = 0, B = 1为正忙状态。还可以另外设置中断请求触发器INTR (为1发出中断请求) 和屏蔽触发器MASK (与INTR配合使用 (8.4)))\par
		\section{程序查询方式}
			\subsection{程序查询流程}
				当I/O设备较多时,优先级查询 (轮询)。通常要执行三条指令:测试指令 (测试设备是否准备就绪)、传送指令 (处理I/O请求)、转移指令 (未就绪,转移到下一个设备询问)。程序流程如下图所示:\par
				\includegraphics[scale = 0.25]{../pictures/program_ask.png}\par
				\includegraphics[scale = 0.23]{../pictures/Program_ask_2.png}\par
				该方式必须将I/O查询程序插入到现有到程序中,故此方式包括以下几项:
				\begin{enumerate}[label=\circled{\arabic*}]
					\item 由于需要占用CPU的寄存器,所以需要先保护起来
					\item 由于转送是一批数据,因此需要先设置I/O设备与主机交换的计数值
					\item 设置欲传送数据在主存缓冲区的首地址
					\item CPU启动I/O设备
					\item 将I/O设备中的设备状态标志取至CPU并测试设备是否就绪 (在设备接口电路里面输入缓冲满或者输出缓冲空),没有就等着
					\item CPU执行I/O指令,读或者写一个数据,并将设备的状态标志\textbf{复位}
					\item 修改主存地址
					\item 修改计数值 (原码 --;负数的补码 ++)
					\item 判断计数值,若不为0,重新启动外设 (即跳转至\circled{6})
					\item 结束I/O传送,继续执行现有程序
				\end{enumerate}
			\subsection[]{程序查询方式的接口电路\footnote{唐书P192,电子书204}}
		\section{程序中断方式}
			\textit{有关中断的详细内容在第八章}\par
			\subsection{程序中断方式的接口电路}
				设备部件:\par
				1. 中断请求触发器和中断屏蔽触发器\\
					当且仅当设备准备就绪 (D = 1),且该设备未被屏蔽 (MASK = 0),CPU的中断查询信号可以 (INTR = 1)\par
				2. 排队器\\
					就I/O中断而言,速度越高的设备优先级越高。硬件排队器可以在CPU里面设置一个计数器 (图8.25),也可以用链式排队器,如图所示:\par
					\includegraphics[scale = 0.25]{../pictures/linked_queue_in_program_break.png}\par
					当各个中断源均无中断请求时,各个$\overline{\mathbf{INTR_i}}$为高电平,其$\mathbf{INTP_i}'$。当某个中断源提出中断请求时,就会迫使优先级比他低的中断源的INTP变为低电平,从而阻止其中断请求\par
				3. 中断向量地址形成部件 (设备编码器)\\
					中断向量形成部件的输入是来自排队器的输出$\mathbf{INTP}_1$,\dots$\mathbf{INTP_n}$,而其输出则是一个中断向量 (二进制代码表示),其实质是一个编码器。这里必须分清\textbf{向量地址}和\textbf{中断服务程序的入口地址} ({\sffamily{都在主存里面}}) 是两个不同的概念,他们之间的关系类似于一个哈希映射,如图所示:\par
					\includegraphics[scale = 0.23]{../pictures/Device_Number_hash.png}\par
				4. 程序中断方式接口电路的基本组成 (如图)\par
					\includegraphics[scale = 0.2]{../pictures/Program_Interrupt_Interface.png}\par
				各问题的处理方式:
				({\ttfamily\textit {来自}}{\sffamily PPT})\par
				\begin{enumerate}[label={\sffamily(\alph{*})}]
					\item 各中断源如何向CPU提出请求 ?(INTR, INTA)
					\item 各中断源同时提出请求怎么办 ? (中断判优-程序查询,硬件排队-集中、分布)
					\item CPU什么条件、什么时间、以什么方式响应中断 ? (指令周期结束,中断隐指令)
					\item 如何保护现场 ?(中断隐指令断点、ISR、堆栈)
					\item 如何寻找入口地址 ?(硬件向量,软件查询)
					\item 如何恢复现场,如何返回 ?(中断隐指令断点、ISR、堆栈)
					\item 处理中断的过程中又出现新的中断怎么办 ? (多重中断,可屏蔽,设置屏蔽字)
				\end{enumerate}
			\subsection[]{程序中断方式的I/O中断处理过程}
				% \begin{markdown}
				\begin{enumerate}
					\item CPU相应中断的时间和条件\\
					CPU响应I/O设备提出中断请求的条件是必须满足CPU中的允许中断触发器 (互斥锁) 为\textsf{1}。该触发器可用开中断指令置位,也可以用关中断指令或者硬件使其自动复位。由于I/O准备就绪的时间 (D = 1) 是随机的,而CPU是在统一的时刻 (EX段结束前) 向接口发中断查询信号,以获取I/O的中断请求。故CPU响应中断的时间一定是在\texttt{\textit{执行阶段的结束时刻}}
					\item I/O中断处理流程 (以输入设备为例)\\
					唐书P198,电子书P211
					\item 中断服务程序的流程
					\begin{enumerate}
						\item 保护现场
						\item 中断服务
						\item 恢复现场
						\item 中断返回
					\end{enumerate}
				\end{enumerate}
				% \end{markdown}
		\section{DMA方式}
			\textit{特别 适用于高速I/O}\par
			在DMA方式中,数据传送前后的准备和处理工作,由CPU上的管理程 序负责,DMA控制器仅负责数据传送工作\par
			\subsection{DMA与主存交换数据时采用如下三种方法:}\par
			\begin{enumerate}
				\item 停止CPU访问主存 (在外设要求传送的时候CPU就放弃,然后一直stall)
				\item 周期挪用/周期窃取 (读书人的事);当DMA要求访问主存时,分为如下三种情况\par
				\begin{enumerate}
					\item available (CPU不在也不打算访存):直接访问内存
					\item busy (CPU正在访存):等着
					\item 访存冲突:CPU在执行访存指令的过程中插入了DMA请求,并挪用了一、二个存取周期,使CPU延缓了一、二个存取周期再访问主存
				\end{enumerate}
				\item DMA与CPU交替访问 (Round-Robin)
			\end{enumerate}
			访存冲突的原因其实是对同一个目标的写冲突
	\chapter[]{中断系统\footnote{唐书第八章第四节、COD Ch4.9}}
		\section{理论部分 (唐书)}
			\subsection{概述}
				\begin{enumerate}
					\item 引起中断的各种因素\par
					\begin{enumerate}
						\item 人为设置的中断\\
						一般称为自愿中断,执行策略如图所示:\par
						\centerline{\includegraphics[height=3cm,width=5cm]{../pictures/Volunteer_Interrupt.png}}
						\centerline{自愿中断}\par
						\item 程序性事故,可能发生的异常如图所示 (常见的)\par
						\centerline{\includegraphics[scale = 0.2]{../pictures/Exceptions.png}}
						\centerline{异常}\par
						\item 硬件故障
						\item I/O设备请求中断
						\item 外部事件 (用户通过键盘来发出中断,如ctrl+C/D/Z)
					\end{enumerate}
					\item 中断系统需要解决的问题
					\begin{enumerate}
						\item 各中断源如何向CPU提出中断请求
						\item 当多个中断源同时提出请求时,中断系统对其作出响应仲裁
						\item CPU在什么条件、什么时候、以什么方式来响应中断
						\item CPU在响应中断之后如何保护现场
						\item CPU响应中断后,如何停止原程序的执行而转入中断服务程序的入口地址
						\item 中断处理结束后,CPU如何恢复现场并返回到原程序的间断处
						\item 若在中断处理过程中又出现了新的中断请求,CPU应该如何处理
					\end{enumerate}
				\end{enumerate}
			\subsection{中断请求标记和中断判优逻辑}
\end{document}
